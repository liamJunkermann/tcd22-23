\documentclass{article}
\usepackage[utf8]{inputenc}

\usepackage{fancyhdr} 
\usepackage{lastpage} 
\usepackage{extramarks} 
\usepackage{graphicx,color}
\usepackage{anysize}
\usepackage{amsmath}
\usepackage{natbib}
\usepackage{caption}
\usepackage{listings}
\usepackage{listings-golang} % import this package after listings
\usepackage{float}
\usepackage{url}
\usepackage{listings}
\usepackage{lmodern}
\usepackage{anyfontsize}
\usepackage[svgnames]{xcolor}
\usepackage[colorlinks=true, linkcolor=Black, urlcolor=Black]{hyperref}

\textwidth=6.5in
\linespread{1.0} % Line spacing
\renewcommand{\familydefault}{\sfdefault}

\newcommand{\includecode}[4]{\lstinputlisting[float,floatplacement=H, caption={[#1]#2}, captionpos=b, frame=single, label={#3}]{#4}}

%% includescalefigure:
%% \includescalefigure{label}{short caption}{long caption}{scale}{filename}
%% - includes a figure with a given label, a short caption for the table of contents and a longer caption that describes the figure in some detail and a scale factor 'scale'
\newcommand{\includescalefigure}[5]{
\begin{figure}[H]
\centering
\includegraphics[width=#4\linewidth]{#5}
\captionsetup{width=.8\linewidth} 
\caption[#2]{#3}
\label{#1}
\end{figure}
}

%% includefigure:
%% \includefigure{label}{short caption}{long caption}{filename}
%% - includes a figure with a given label, a short caption for the table of contents and a longer caption that describes the figure in some detail
\newcommand{\includefigure}[4]{
\begin{figure}[H]
\centering
\includegraphics{#4}
\captionsetup{width=.8\linewidth} 
\caption[#2]{#3}
\label{#1}
\end{figure}
}

%% Code formatting:
\usepackage{xcolor}
\definecolor{light-gray}{gray}{0.95}
\newcommand{\code}[1]{\colorbox{light-gray}{\texttt{#1}}}

\newcommand{\codelisting}[2]{
  \code{#1}
  \label{code:#2}
  \lstinputlisting{../src/#1}
  \vspace{2em}
}

\newcommand{\coderef}[1]{
  \hyperref[code:#1]{\code{#1}}
}


\lstset{ % add your own preferences
    frame=single,
    basicstyle=\footnotesize,
    keywordstyle=\color{red},
    numbers=left,
    numbersep=5pt,
    showstringspaces=false, 
    stringstyle=\color{blue},
    tabsize=4,
    language=Golang % this is it !
}

%%------------------------------------------------
%% Parameters
%%------------------------------------------------
% Set up the header and footer
\pagestyle{fancy}
\lhead{\authorName} % Top left header
\chead{\moduleCode\ - \assignmentTitle} % Top center header
\rhead{\firstxmark} % Top right header
\lfoot{\lastxmark} % Bottom left footer
\cfoot{} % Bottom center footer
\rfoot{Page\ \thepage\ of\ \pageref{LastPage}} % Bottom right footer
\renewcommand\headrulewidth{0.4pt} % Size of the header rule
\renewcommand\footrulewidth{0.4pt} % Size of the footer rule

\setlength\parindent{0pt} % Removes all indentation from paragraphs
\newcommand{\assignmentTitle}{Project 1 - A Web Proxy Server}
\newcommand{\moduleCode}{CSU34031} 
\newcommand{\moduleName}{Advanced Computer Networks} 
\newcommand{\authorName}{Liam Junkermann} 
\newcommand{\authorID}{19300141}
\newcommand{\reportDate}{\today}
\renewcommand{\abstractname}{Introduction}

\title{
    \vspace{-1in}
    \begin{figure}[!ht]
    \flushleft
    \includegraphics[width=0.4\linewidth]{reduced-trinity.png}
    \end{figure}
    \vspace{-0.5cm}
    \hrulefill \\
    \vspace{1cm}
    \textmd{\textbf{\moduleCode\ \moduleName}}\\
    \textmd{\textbf{\assignmentTitle}}\\
    \textmd{\authorName\ - \authorID}\\
    \textmd{\reportDate}\\
    \vspace{0.5cm}
    \hrulefill \\
}
\date{}
\author{}

\begin{document}
    \lstset{language=bash, float=h, captionpos=b, frame=single, numbers=left, numberblanklines=false, numberstyle=\tiny, numbersep=1mm, framexleftmargin=3mm, xleftmargin=5mm, aboveskip=3mm, breaklines=true}
    \captionsetup{width=.8\linewidth} 

    \maketitle
    \begin{abstract}
      The goal of this assignment was to build a web proxy with the following details:
      \begin{enumerate}
        \item Respond to HTTP \& HTTPS requests and display each request on a management console
        \item Handle websocket connections
        \item Dynamically block selected URLs via the management console
        \item Efficiently cache HTTP request locally
        \item Handle multiple requests through threading
      \end{enumerate}
    \end{abstract}
    \tableofcontents
    \newpage
    
    \section{Design and Implementation}
    \label{sec:Design}
    This webproxy has 3 main parts: the proxy server (handled in \coderef{proxyserver.go}), the cache (a file-based implementation built in \coderef{filecache.go}), the web dashboard which manages blocking of urls (run from \coderef{dashboard.go}, and \coderef{dynamicblock.go}). 
    The proxy server handles the proxy connections by either forwarding the necessary https \code{CONNECT} requests and creating the appropriate tunnels, or repeating, caching, and responding to HTTP requests. 
    The cache is managed through responses being saved to files with the request content included. The hashes are saved in memory along with timing and bandwidth data (this data is lost when the server is stopped). Finally, a web dashboard was created to manage the URLs and blocking selected URLs, currently the web dashboard needs to request the list of proxied endpoints, but a websocket approach could be used going forward to allow for streaming of URLs to the web management console.
    These pieces are combined into a final executable (\coderef{final.go}) which can be deployed to a docker container or directly to an active server. The architecture of this proxy server does not add any headers 
    \newpage
    \section{Code Listings}
    \label{sec:codeListing}
    \codelisting{proxyserver.go}{proxyserver.go}
    \codelisting{pkg/cache/filecache/filecache.go}{filecache.go}
    \codelisting{pkg/dashboard/dashboard.go}{dashboard.go}
    \codelisting{pkg/dynamicblock/dynamicblock.go}{dynamicblock.go}
    \codelisting{cmd/final/final.go}{final.go}
\end{document}
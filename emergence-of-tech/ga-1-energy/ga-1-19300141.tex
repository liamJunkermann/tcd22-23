\documentclass{article}
\usepackage[utf8]{inputenc}

\usepackage{fancyhdr} 
\usepackage{lastpage} 
\usepackage{extramarks} 
\usepackage{graphicx,color}
\usepackage{anysize}
\usepackage{amsmath}
\usepackage{natbib}
\usepackage{caption}
\usepackage{float}
\usepackage{url}
\usepackage{listings}
\usepackage[svgnames]{xcolor}
\usepackage[colorlinks=true, linkcolor=Black, urlcolor=Black]{hyperref}
\usepackage[small]{titlesec}
\usepackage{mhchem}
\usepackage{linegoal}


\textwidth=6.5in
\linespread{1.0} % Line spacing
\renewcommand{\familydefault}{\sfdefault}

% \titleformat{\section}
% {\normalfont\bfseries}
% {\thesection.}{0.75em}{}

\titlespacing{\section}{0pt}{0pt}{0pt}

%% includescalefigure:
%% \includescalefigure{label}{short caption}{long caption}{scale}{filename}
%% - includes a figure with a given label, a short caption for the table of contents and a longer caption that describes the figure in some detail and a scale factor 'scale'
\newcommand{\includescalefigure}[5]{
\begin{figure}[H]
\centering
\includegraphics[width=#4\linewidth]{#5}
\captionsetup{width=.8\linewidth} 
\caption[#2]{#3}
\label{#1}
\end{figure}
}

%% includefigure:
%% \includefigure{label}{short caption}{long caption}{filename}
%% - includes a figure with a given label, a short caption for the table of contents and a longer caption that describes the figure in some detail
\newcommand{\includefigure}[4]{
\begin{figure}[H]
\centering
\includegraphics{#4}
\captionsetup{width=.8\linewidth} 
\caption[#2]{#3}
\label{#1}
\end{figure}
}

%%------------------------------------------------
%% Parameters
%%------------------------------------------------
% Set up the header and footer
\pagestyle{fancy}
\lhead{\authorName} % Top left header
\chead{\moduleCode\ - \assignmentTitle} % Top center header
\rhead{\firstxmark} % Top right header
\lfoot{\lastxmark} % Bottom left footer
\cfoot{} % Bottom center footer
\rfoot{Page\ \thepage\ of\ \pageref{LastPage}} % Bottom right footer
\renewcommand\headrulewidth{0.4pt} % Size of the header rule
\renewcommand\footrulewidth{0.4pt} % Size of the footer rule

\setlength\parindent{0pt} % Removes all indentation from paragraphs
\newcommand{\assignmentTitle}{Group Assignment 1 - Energy}
\newcommand{\moduleCode}{TEU00041} 
\newcommand{\moduleName}{Emergence of Technologies} 
\newcommand{\authorName}{Liam Junkermann} 
\newcommand{\authorID}{19300141}
\newcommand{\reportDate}{\today}
\renewcommand{\abstractname}{Introduction}

\title{
    \vspace{-1in}
    \begin{figure}[!ht]
    \flushleft
    \includegraphics[width=0.4\linewidth]{reduced-trinity.png}
    \end{figure}
    \vspace{-0.5cm}
    \hrulefill \\
    \vspace{1cm}
    \textmd{\textbf{\moduleCode\ \moduleName}}\\
    \textmd{\textbf{\assignmentTitle}}\\
    \textmd{\authorName\ - \authorID}\\
    \textmd{\reportDate}\\
    \vspace{0.5cm}
    \hrulefill \\
}
\date{}
\author{}

\begin{document}
    \lstset{language=bash, float=h, captionpos=b, frame=single, numbers=left, numberblanklines=false, numberstyle=\tiny, numbersep=1mm, framexleftmargin=3mm, xleftmargin=5mm, aboveskip=3mm, breaklines=true}
    \captionsetup{width=.8\linewidth} 

    \maketitle
    \begin{abstract}
        Your group has been tasked with developing a strategy for dealing with the short-term acute energy crisis that Ireland and the rest of Europe is facing. Yours brief is to concentrate on measures that could mitigate the crisis for the next three years, with a special focus on the coming winter. Your brief recognises that there are two potential but related problems: first, the risk of interruption in the supply of energy, particular natural gas; and second, the problem of increased energy costs. Your primary objective is to explore methods of ensuring that Irish residents have supplies of electricity and heating but should also consider the economic aspects of the problem. 
    \end{abstract}
    \newpage

    \section{Short-term Mitigations}
    {\scriptsize\textbf{Detail your proposed mitigation measures. These measures should be classified into two categories: actions that can be implemented within the next three months (winter 2022/23), and actions that can be put in place for the winters of 2023/24 and 2024/25.}}
    \vspace{10pt}

    To address the energy crisis in the short-term (winter 2022/23) the following initiatives can be employed:
    \begin{description}
        \item [Increased the importation of Liquid Natural Gas (LNG) to supplement existing supplies.] The recent depletion of the Corrib Gas Field has placed increased strain on the Irish gas supply. LNG offers slightly cleaner energy production than equivalent coal plants, with increased efficiency and reduced \ce{CO2} emissions. With approximately 15\footnote{\url{https://openinframap.org/stats/area/Ireland/plants}} gas-powered energy plants in Ireland, greater importation of LNG in the short term can help boost the grid when renewable sources are not producing enough power.
        \item [Improving energy efficiency in homes and public buildings.] The national upgrade of smart meters will enable all buildings in Ireland to have a better understanding of their energy consumption and times at which energy consumption is increased. Using this information users can optimise when high energy load activities, such as cooking and laundry are done to minimise strain on the grid. Additionally, energy companies can encourage this behaviour by reducing costs during low load times, with further tracking and accountability provided by these smart meters.
    \end{description}

    For the next two winters (2023/24 and 2024/25) initiatives such as the following can be employed to continue to ease the strain on the energy grid:
    \begin{description}
        \item[Encouraging installation and use of renewable sources for homes and businesses.] A very easy way to reduce the general grid strain is to encourage, by setting up government and provider programmes and initiatives, the installation of renewable energy production on site at users' homes and places of business. As the technology becomes more efficient homes will have a lower base strain on the grid, and with the installation of batteries be able to further reduce their energy draw when renewable sources are not producing.
        \item[Short term battery investments] As the amount of renewable based energy production sites increases, larger batteries will be needed to ensure that all the energy produced by these sites is stored and used. As a short-term mitigation, quickly increasing the battery capacity of the grid will also allow for the temporary storage of power to help alleviate the risk of brown-outs during peak usage.
    \end{description}

    \section{Risks}
    {\scriptsize\textbf{For each of your proposed actions, identify any risks that could prevent the action being implemented or being effective, plus any further steps that could be take to mitigate these risks.}}
    \vspace{10pt}

    The most salient risk with the above described plans is the clear financial impact across the country. Many of these mitigations require government or organisational investment, whether by immediate cash spend or in reduced tax and future revenue in the case of some incentives, to effectively carry out demand management and initiatives such as the installation of smart meters. Increasing LNG importation can also result in a general increase in price globally, adding to the financial strain felt by the population throughout the current cost of living crisis. Given the current financial and economic climate coming out of the COVID-19 Pandemic there may be a need for the government to refinance some debt to make these investments, and continue to encourage and incentivise those who are able to make personal investments in renewable technologies to reduce the overall grid strain.
    \vspace{10pt}


    \section{Long-Term Strategy}
    {\scriptsize\textbf{In parallel you have been asked to develop a medium to long term strategy for energy provision in Ireland. The overall aims of the strategy will be to: reduce carbon emissions to a minimum, ensure robustness of supply and ensure cost competitiveness. Your strategy should cover all forms of energy usage and should address the issues of energy storage and transport energy demands explicitly.}}
    \vspace{10pt}

    Many of the long-term strategies to reduce the risk of energy crisis build on the medium-term mitigations such as investment in renewable power solutions. The national scale investment in renewable technologies for electricity generation will also massively reduce the carbon footprint of energy production in Ireland. With initiatives like the National Renewable Energy Action Plan (NREAP)\footnote{\url{https://www.ifa.ie/wp-content/uploads/2020/08/2013-National-Renewable-Energy-Action-Plan-2010.pdf}} already in action, the implementation of such technologies and investment will be crucial to reducing the reliance on imported fuels and carbon emission heavy energy production. This will result in lower cost of energy in the long term, once the initial investment has matured. In order to move to a mostly renewable platform, the necessary investment into energy storage solutions must be made as well, particularly in the winter months when a renewable source such as solar power may not be as effective. As future energy generation technologies such as hydrogen and safer nuclear fission become more accessible and cheaper, the necessary investment in the implementation of these technologies can also be employed to reduce the effect of the energy crisis and carbon emissions simultaneously. Each of these solutions relies on the proposal and adoption of policy from a government level, followed by the necessary incentives to execute and develop the necessary sites and technologies.
    
\end{document}
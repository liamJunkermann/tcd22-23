\documentclass{article}
\usepackage[utf8]{inputenc}

\usepackage{fancyhdr} 
\usepackage{lastpage} 
\usepackage{extramarks} 
\usepackage{graphicx,color}
\usepackage{anysize}
\usepackage{amsmath}
\usepackage{natbib}
\usepackage{caption}
\usepackage{float}
\usepackage{url}
\usepackage{listings}
\usepackage[svgnames]{xcolor}
\usepackage[colorlinks=true, linkcolor=Black, urlcolor=Black]{hyperref}
\usepackage[small]{titlesec}
\usepackage{mhchem}
\usepackage{linegoal}


\textwidth=6.5in
\linespread{1.0} % Line spacing
\renewcommand{\familydefault}{\sfdefault}

% \titleformat{\section}
% {\normalfont\bfseries}
% {\thesection.}{0.75em}{}

\titlespacing{\section}{0pt}{0pt}{0pt}

%% includescalefigure:
%% \includescalefigure{label}{short caption}{long caption}{scale}{filename}
%% - includes a figure with a given label, a short caption for the table of contents and a longer caption that describes the figure in some detail and a scale factor 'scale'
\newcommand{\includescalefigure}[5]{
\begin{figure}[H]
\centering
\includegraphics[width=#4\linewidth]{#5}
\captionsetup{width=.8\linewidth} 
\caption[#2]{#3}
\label{#1}
\end{figure}
}

%% includefigure:
%% \includefigure{label}{short caption}{long caption}{filename}
%% - includes a figure with a given label, a short caption for the table of contents and a longer caption that describes the figure in some detail
\newcommand{\includefigure}[4]{
\begin{figure}[H]
\centering
\includegraphics{#4}
\captionsetup{width=.8\linewidth} 
\caption[#2]{#3}
\label{#1}
\end{figure}
}

%%------------------------------------------------
%% Parameters
%%------------------------------------------------
% Set up the header and footer
\pagestyle{fancy}
\lhead{\authorName} % Top left header
\chead{\moduleCode\ - \assignmentTitle} % Top center header
\rhead{\firstxmark} % Top right header
\lfoot{\lastxmark} % Bottom left footer
\cfoot{} % Bottom center footer
\rfoot{Page\ \thepage\ of\ \pageref{LastPage}} % Bottom right footer
\renewcommand\headrulewidth{0.4pt} % Size of the header rule
\renewcommand\footrulewidth{0.4pt} % Size of the footer rule

% \setlength\parindent{0pt} % Removes all indentation from paragraphs
\newcommand{\assignmentTitle}{Final Essay}
\newcommand{\moduleCode}{TEU00402} 
\newcommand{\moduleName}{How to Live Long And Prosper} 
\newcommand{\authorName}{Liam Junkermann} 
\newcommand{\authorID}{19300141}
\newcommand{\reportDate}{\today}
\renewcommand{\abstractname}{Introduction}

\title{
    \vspace{-1in}
    \begin{figure}[!ht]
    \flushleft
    \includegraphics[width=0.4\linewidth]{reduced-trinity.png}
    \end{figure}
    \vspace{-0.5cm}
    \hrulefill \\
    \vspace{1cm}
    \textmd{\textbf{\moduleCode\ \moduleName}}\\
    \textmd{\textbf{\assignmentTitle}}\\
    \textmd{\authorName\ - \authorID}\\
    \textmd{\reportDate}\\
    \vspace{0.5cm}
    \hrulefill \\
}
\date{}
\author{}

\begin{document}
    \lstset{language=bash, float=h, captionpos=b, frame=single, numbers=left, numberblanklines=false, numberstyle=\tiny, numbersep=1mm, framexleftmargin=3mm, xleftmargin=5mm, aboveskip=3mm, breaklines=true}
    \captionsetup{width=.8\linewidth} 

    \maketitle
    \newpage

Aging has many profound impacts on the body affecting every aspect of life. These changes can be affected by a myriad of environmental and genetic factors, but no matter what, cells continue to age regardless of the body they are in. These cellular level changes start with minor changes in the speed and efficiency of the cell replication, resulting in more substantial changes to how the body operates impacting how quickly recovery from exertion can occur, the efficiency of a number of vital organs, and increasing the risk and magnitude of any given disease or sickness. The reduced organ efficiency results in the body being less equipped to clear toxins throughout the body resulting in increased build up and strain on the cardiovascular and immune systems. There are many environmental factors that can affect the aging of cells, including if a person is physically active, has good nutrition, engages in some kind of calorie restriction diet, their social engagement, and many more factors. A combination of these cellular aging characteristics and mitigations can result in a longer life with higher quality of life, and important factor particularly in old age. Despite the mitigations which can be made, the result of cellular aging is final and has a considerable effect on the aging of the rest of the body.

Cellular changes in the body drive many of the negative impacts of aging, and are the source of many of those changes. Many factors lead to these cellular changes, including genomic and DNA breakdown and the resulting consequences. As cells grow and age, the body naturally replicates and replaces individual cells as needed. However, throughout this process a number of factors can affect the replication process. External factors such as exposure to certain physical, chemical or environmental elements can cause damage to the DNA chain which will be replicated. Internally there is constantly the risk of DNA replication errors. These behaviours result in minor changes and damage to extensive DNA chains which, over time, begin to breakdown, mutate, and drive many of the risk factors associated with aging such as reduced organ function and increased risk for cancers. One of the results of this DNA damage is the shortening of telomeres. Telomere shortening has been extensively studied and has shown to have a significant impact on the aging of cells due to their ability to protect from the loss of base pairs in DNA replication. Interventions taken to manage telomere shortening have been shown to reduce the risk of a variety of chronic conditions including hypertension, stroke, and some cancers. The effects of cellular and DNA changes over time have a profound impact on aging and are a significant contributor to other changes which cumulatively lead to the increased vulnerability to death which exemplifies aging.


Research into the cellular-level impact of aging has afforded a greater understanding of the underlying mechanisms which result in reduced effectiveness and function of more general body systems. The immune system sees a profound impact as a result of the damage done by aging to cells. As people, and therefore their cells, age, their immune systems become less agile and able to handle incoming threats to their health. Due to decreased function from damaged cells, there is a reduced production of B and T cells in bone marrow. These cells are crucial in mobilising and protecting the body from foreign invaders which can result in autoimmune diseases, further debilitating an already compromised system. Immunosenescence is further affected by the reduced function of lymphatic organs, damage caused by a lifetime of fighting infections and diseases in conjunction with the reduced function on a cellular level. Lifelong accrual of cellular damage is a clear contributor to immunosenescence through reduced function of bone marrow to produce key immune response cells, and the reduced efficacy of the lymphatic system as a result of cellular level damage.


Basic cardiovascular function is essential to any step of life, particularly to live with even a passable quality of life. It is well known that age-related changes can negatively impact the various systems in our bodies, leading to increased susceptibility to disease. The aging of cells and the resulting inefficiency of organs to handle toxins are among the most significant factors that impact the cardiovascular system. As a result of the "inflamm-aging" associated with immunosenescence, the hardening of arteries in conjunction with the increase in plaque deposits along the artery walls lead to higher risk of heart disease, strokes, and other cardiovascular conditions and diseases. The effects of "inflamm-aging" are particularly relevant to the cardiovascular system, as it can lead to arterial stiffness, atherosclerosis, and increased oxidative stress. All of these factors contribute to an increased risk of cardiovascular events. The harder arterial walls resulting from the build-up of plaque can also result in higher blood pressure, adding strain to an already inefficient cardiovascular system and the organs that rely on it. The added strain on the cardiovascular system can be particularly detrimental to those with pre-existing cardiovascular conditions regardless of any aging. It is essential to recognise the impact of age-related changes on the cardiovascular system and take steps to reduce the risk of disease. Maintaining a healthy lifestyle, including regular exercise and a balanced diet, can significantly reduce the risk of cardiovascular disease.


Social engagement and mental health have a significant impact on aging. While these factors may not be directly related to cellular aging, it is important to consider the effect they may have on aging brains and the quality of life. Strong social engagement and mental health can mitigate cognitive decline, which is a standard effect of aging. However, neurodegenerative diseases accelerate this decline, making even basic tasks nearly impossible to achieve alone. This can result in declining mental health, something which people of all ages can struggle with. One of the easiest combatants to this cognitive decline, marked by reduced hippocampal health, is social engagement. This impact of social engagement in community, with people of all ages, is clearly shown in blue zones. These are areas around the world where life expectancy is much greater than the global average. Older people living in these areas consistently engage with other members of their communities from varying backgrounds and age groups. These other community members help take care of the aging population while also allowing the aging group to socialise with and, crucially, mentally engage with others on a day-to-day basis. This behaviour's impact, combined with other dietary, exercise, and social behaviours, has been shown, through many longitudinal studies, to be highly effective in reducing the effect of aging. The impact of social engagement on aging is not only limited to cognitive decline but it also affects mental health. Loneliness and social isolation have been linked to depression, anxiety, and other mental health problems in older adults. On the other hand, social engagement can help older adults build a sense of purpose, boost their self-esteem, and reduce stress levels. Research has shown that socially engaged older adults have a lower risk of developing depression and are less likely to experience cognitive decline. Moreover, social engagement can also have a positive impact on physical health. Studies have found that older adults who are socially engaged are less likely to develop chronic diseases such as hypertension, diabetes, and heart disease. Social engagement can also help older adults to maintain physical functioning and mobility, which can reduce their risk of falls and other physical injuries.


Aging affects many facets of life, driven primarily by the breakdown of cellular and DNA-level processes to continue replicating and building. This gradual aging at the cellular level begins to affect more complex processes resulting in lower efficiency in organs, lower sensitivity to environmental changes such as nutrient intake or infections -- resulting in slower response from the immune system, and an ever-decreasing ability to clear toxins from the body successfully. Each of these factors leads to increased susceptibility to disease and infection, with each such case compromising immunity and shortening the lifespan in an already aging body, further increasing the vulnerability of age. However, despite all the research showing the substantial impact time has on aging cells, there are still interventions which can be used to help slow, and in some cases reverse, the effect of time on aging cells and DNA. These interventions include consistent exercise and resistance training, restricted calorie diets such as intermittent fasting, and some pharmacological interventions to manage specific issues. Additionally, social engagement and mental health are crucial components of healthy aging. Engaging in social activities and maintaining strong social connections can help older adults to maintain their cognitive and physical health, reduce their risk of developing chronic diseases, and improve their overall quality of life. It is essential to encourage and support older adults in engaging in social activities to ensure that they remain socially connected and mentally healthy. While the effects of aging and time on cells being so significant, it is crucial that interventions are employed sooner in life for those looking to improve not only lifespan, but quality of life in old age as well.


\newpage
\bibliographystyle{plainnat}
\nocite{*}
\bibliography{refs}

    
\end{document}